\documentclass{article}
\usepackage[utf8]{inputenc}
\usepackage{geometry}
\usepackage{amsmath}

\geometry{margin = 2cm}

\title{Analytic magnetic field around picture-frame coils}
\author{Alexander Prokopyszyn}
\date{May 2024}

\begin{document}

\maketitle

\section{Magnetic field around a vertical wire}

Consider a vertical wire at the origin with current pointing the in the $\mathbf{\hat{z}}$ direction and extends from $z=z_{min}$ to $z=z_{max}$. Using a formula which is derived in for example:
\begin{verbatim}
https://www.youtube.com/watch?v=BlIEavkDUF8&t=400s
\end{verbatim}
we know that the magnetic field around this wire is given by:
\[\mathbf{B}_{vwire}(R, \phi, z) = \frac{\mu_0I}{4\pi}\frac{1}{R} \left[\sin(\theta_1) + \sin(\theta_2)\right]\mathbf{\hat{\phi}}.\]
Using
\[\sin(\theta_1) = \frac{z_{max}-z}{\sqrt{R^2+(z_{max}-z)^2}},\]
\[\sin(\theta_2) = \frac{z-z_{min}}{\sqrt{R^2+(z-z_{min})^2}},\]
gives
\[\mathbf{B}_{vwire}(R, \phi, z) = \frac{\mu_0I}{4\pi}\frac{1}{R} \left[\frac{z_{max}-z}{\sqrt{R^2+(z_{max}-z)^2}} + \frac{z-z_{min}}{\sqrt{R^2+(z-z_{min})^2}}\right]\mathbf{\hat{\phi}}.\]
Using
\[R = \sqrt{x^2+y^2},\]
\[\mathbf{\hat{\phi}}=\frac{-y\mathbf{\hat{x}}+x\mathbf{\hat{y}}}{\sqrt{x^2+y^2}},\]
we can convert this to Cartesian coordinates to give
\[\mathbf{B}_{vwire}(x, y, z; z_{min}, z_{max}, I) = \frac{\mu_0I}{4\pi}\frac{1}{x^2+y^2} \left[\frac{z_{max}-z}{\sqrt{x^2+y^2+(z_{max}-z)^2}} + \frac{z-z_{min}}{\sqrt{x^2+y^2+(z-z_{min})^2}}\right](-y\mathbf{\hat{x}}+x\mathbf{\hat{y}}).\]
Let
\[A_{vwire}(x, y, z; z_{min}, z_{max}, I) = \frac{\mu_0I}{4\pi}\frac{1}{x^2+y^2}
\left[\frac{z_{max}-z}{\sqrt{x^2+y^2+(z_{max}-z)^2}} + \frac{z-z_{min}}{\sqrt{x^2+y^2+(z-z_{min})^2}}\right].\]
In vector notation $\mathbf{B}_{vwire}$ is given by
\[\boxed{\mathbf{B}_{vwire}(x, y, z; z_{min}, z_{max}, I) = A_{vwire}(x, y, z; z_{min}, z_{max}, I)
\begin{pmatrix}
    -y \\ x \\ 0
\end{pmatrix}}\]

\section{Magnetic field around a horizontal wire}

We can use the formula for the magnetic field around a vertical wire to calculate the
 magnetic field around a wire that points
in the $\mathbf{\hat{R}}_k = \cos(\phi_k)\mathbf{\hat{x}}+\sin(\phi_k)\mathbf{\hat{y}}$ direction
 and extends from $R=R_{inner}$ to $R_{outer}$ at $z=z_0$.
 We do this by taking a vertical wire which points in the $z$-direction
 and is centred at the then apply a 90 degree rotation around the $y$-axis,
  and then a $\phi_k$ rotation around the $z$-axis.

Using the formula for the rotation matrix about the $y$ and $z$ axes
\[\text{R}_y(\theta) = \begin{pmatrix}
    \cos(\theta) & 0 & \sin(\theta) \\
    0 & 1 & 0 \\
    -\sin(\theta) & 0 & \cos(\theta)
\end{pmatrix},\]
\[\text{R}_z(\theta) = \begin{pmatrix}
    \cos(\theta) & -\sin(\theta) & 0 \\
    \sin(\theta) & \cos(\theta) & 0 \\
    0 & 0 & 1
\end{pmatrix}.\]
Our full rotation matrix is given by
\[\begin{aligned}
    \text{R}_k&=\text{R}_z(\phi_k)\text{R}_y(\pi/2) \\
    &=\begin{pmatrix}
        \cos(\phi_k) & -\sin(\phi_k) & 0 \\
        \sin(\phi_k) & \cos(\phi_k) & 0 \\
        0 & 0 & 1
    \end{pmatrix}
    \begin{pmatrix}
        0 & 0 & 1 \\
        0 & 1 & 0 \\
        -1 & 0 & 0
    \end{pmatrix} \\
    &= \begin{pmatrix}
        0 & -\sin(\phi_k) & \cos(\phi_k) \\
        0 & \cos(\phi_k) & \sin(\phi_k) \\
        -1 & 0 & 0
    \end{pmatrix}.
\end{aligned} \]
The inverse rotation matrix is given by
\[\begin{aligned}
    \text{R}^{-1}_k &= \text{R}_y(-\pi/2)\text{R}_z(-\phi_k) \\
    &= \begin{pmatrix}
        0 & 0 & -1 \\
        0 & 1 & 0 \\
        1 & 0 & 0
    \end{pmatrix}
    \begin{pmatrix}
        \cos(\phi_k) & \sin(\phi_k) & 0 \\
        -\sin(\phi_k) & \cos(\phi_k) & 0 \\
        0 & 0 & 1
    \end{pmatrix} \\
    &= \begin{pmatrix}
        0 & 0 & -1 \\
        -\sin(\phi_k) & \cos(\phi_k) & 0 \\
        \cos(\phi_k) & \sin(\phi_k) & 1
    \end{pmatrix}.
\end{aligned}\]
Let 
\[\begin{aligned}
    \begin{pmatrix}
        x_k \\ y_k \\ z_k
    \end{pmatrix} &= \text{R}^{-1}_k
    \begin{pmatrix}
        x \\ y \\ z
    \end{pmatrix} \\
    &= \begin{pmatrix}
        -z \\
        y \cos(\phi_k) - x \sin(\phi_k) \\
        x \cos(\phi_k) + y \sin(\phi_k)
    \end{pmatrix}.
\end{aligned}\]

\[\begin{aligned}
\implies
\begin{pmatrix}
    x \\ y \\ z
\end{pmatrix} &= 
\text{R}_k
\begin{pmatrix}
    x_k \\ y_k \\ z_k
\end{pmatrix} \\
&= \begin{pmatrix}
    z_k\cos(\phi_k)-y_k\sin(\phi_k)  \\
    y_k\cos(\phi_k) +z_k\sin(\phi_k) \\
    -x_k
\end{pmatrix}.
\end{aligned}\]
Hence, the formula for the magnetic field due to the upper wires in the 
 picture-frame coils is given by
\[\mathbf{B}_{upper,k}(x,y,z) = \text{R}_k\mathbf{B}_{vwire}(x_k + h/2, y_k, z_k; R_{inner}, R_{outer}, I),\]
where $h$ is the height of the coils.

\section{Magnetic field in picture-frame coils}

The current design has $N_{coil}=16$ picture-frame TF coils. The $k^{\text{th}}$ coil is composed of four wires:
\begin{enumerate}
    \item A vertical wire at $R=R_{outer}$, $\phi=\phi_k=2\pi k/N_{coil}$ with current in the negative $\mathbf{\hat{z}}$ direction and extends from $z=-h/2$ to $z=+h/2$, where $h$ is the height of the TF coil.
    \item  A vertical wire at $R=R_{inner}$, $\phi=\phi_k$ with current in the positive $\mathbf{\hat{z}}$ direction and extends from $z=-h/2$ to $z=+h/2$.
    \item A horizontal wire with current which points in the $\mathbf{\hat{R}}_k = \cos(\phi_k)\mathbf{\hat{x}}+\sin(\phi_k)\mathbf{\hat{y}}$ direction and extends from $R=R_{inner}$ to $R_{outer}$ at $z=h/2$.
    \item A horizontal wire with current that points in the negative $\mathbf{\hat{R}}_k$ direction and extends from $R=R_{inner}$ to $R_{outer}$ at $z=-h/2$.
\end{enumerate}
We can model the magnetic field from the vertical wires with 
\[\mathbf{B}_{inner,k}(x,y,z) = \mathbf{B}_{vwire}(x-R_{inner}\cos\phi_k, y-R_{inner}\sin{\phi_k}, z; -h/2, h/2, I),\]
\[\mathbf{B}_{outer,k}(x,y,z) = \mathbf{B}_{vwire}(x-R_{outer}\cos\phi_k, y-R_{outer}\sin{\phi_k}, z; -h/2, h/2, -I).\]
We can model the horizontal wires using:
\[\mathbf{B}_{upper,k}(x,y,z) = \text{R}_k\mathbf{B}_{vwire}(x_k + h/2, y_k, z_k; R_{inner}, R_{outer}, I),\]
\[\mathbf{B}_{lower,k}(x,y,z) = \text{R}_k\mathbf{B}_{vwire}(x_k - h/2, y_k, z_k; R_{inner}, R_{outer}, -I).\]
Hence, the full magnetic field from all the TF coils is given by
\[\boxed{\mathbf{B}(x, y, z) = \sum_{k=0}^{N_{coil}-1}\mathbf{B}_{inner,k} + \mathbf{B}_{outer,k} + \mathbf{B}_{lower,k} + \mathbf{B}_{upper,k}.}\]






\end{document}
